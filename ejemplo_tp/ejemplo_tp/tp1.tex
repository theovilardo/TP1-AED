\documentclass[10pt,a4paper]{article}

\input{AEDmacros}
\usepackage{caratula} % Version modificada para usar las macros de algo1 de ~> https://github.com/bcardiff/dc-tex


\titulo{Descripci\'on del tp}
\subtitulo{Subtítulo del tp}

\fecha{\today}

\materia{Materia de la carrera}
\grupo{Grupo 42}

\integrante{Nabot, Teo}{996/22}{teonabot@gmail.com}
\integrante{Santillan, Lautaro}{370/22}{lautisantil@gmail.com}
\integrante{Vilardo, Theo}{743/22}{theovilardo@gmail.com}
\integrante{nico pone tu apellido, Nicolas}{004/01}{maildenico@dominio.com}
% Pongan cuantos integrantes quieran

% Declaramos donde van a estar las figuras
% No es obligatorio, pero suele ser comodo
\graphicspath{{../static/}}

\begin{document}

\maketitle

\section{Ejemplo de sección}

\subsection{Macros de la cátedra para especificar}

%ejemplo proc
% \begin{proc}{nombre}{\In paramIn : \nat, \Inout paramInout : \TLista{\ent}}{tipoRes}
% 	%    \modifica{parametro1, parametro2,..}
% 	\requiere{expresionBooleana1}
% 	\asegura{expresionBooleana2}
% 	\aux{auxiliar1}{parametros}{tipoRes}{expresion}
% 	\pred{pred1}{parametros}{expresion} 
% \end{proc}

\begin{proc}{redistribucionDeLosFrutos}{\In recursos : \TLista{\float}, \In cooperan : \TLista{\bool}}{\TLista{\float}}
	\requiere{|recursos| = |cooperan|}
	\asegura{res = fondoMonetario(recursos, cooperan) / |recursos|}
	\aux{fondoMonetario}{\In recursos : \TLista{\float}, \In cooperan : \TLista{\bool}}{\float}{\sum_{i=0}^{n} {\IfThenElse{cooperan[i]}{recursos[i]}{0}}}
\end{proc}

\begin{proc}{trayectoriaExtrañaEscalera}{\In trayectoria : \TLista{\float}}{\bool}
	\requiere{|recursos| = |cooperan|}
	\asegura{res = {\existe{i}{\nat}{0 <= i < |trayectoria| \yLuego trayectoria[i] > trayectoria[i+1] \wedge trayectoria[i] > trayectoria[i-1]}}}
\end{proc} %no encontré el símbolo existe unico ;;  hay q bver como hacer lo de los bordes porque se indefine ;; caso vacío?

\begin{minipage}[t]{19cm}
\begin{proc}{trayectoriaDeLosFrutosIndividualesALargoPlazo}{\Inout trayectorias : \TLista{\TLista{\mathds{R}}}, \In cooperan : \TLista{\bool}, \In apuestas : \TLista{\TLista{\mathds{R}}}, \In pagos : \TLista{\TLista{\mathds{R}}}, \In eventos : \TLista{\TLista{\mathds{N}}}}{}
	%    \modifica{parametro1, parametro2,..}
	\requiere{faltaAgregarlos}
	\asegura{%
	\paraTodo[unalinea]{j}{\ent}{0 \leq j < \left| trayectorias |\right} \implica {trayectorias[j][0] = old{(trayectorias[j][0])}}
	} 
	\asegura{%
	\paraTodo[unalinea]{j, r}{\ent}{0 \leq j < trayectorias}{\implicaLuego
	(0 < r < eventos[j])} \implica trayectorias[j][r] = 
	calcularRecursos(
		trayectorias[j][r - 1],
		cooperan[j],
		pagos[j][eventos[j][r - 1]],
		apuestas[j][r - 1],
		calcularFondoComun(trayectoria, cooperan, apuestas, pagos, eventos, r)),
		| trayectorias |
	}
\end{proc}
\end{minipage}

%aqui iria un espacio
\vspace{1cm}

\begin{minipage}[t]{19cm}
\aux{calcularFondoComun}{trayectorias : \TLista{\TLista{\mathds{R}}}, cooperan : \TLista{\bool}, apuestas : \TLista{\TLista{\mathds{R}}}, pagos : \TLista{\TLista{\mathds{R}}}, eventos : \TLista{\TLista{\mathds{N}}}}{\float}{%
	%\begin{equation}
	\sum\limits_{i=0}^{| trayectorias | - 1} \IfThenElse{cooperan[i] = true}{trayectorias[i][r - 1] * apuestas[i][eventos[i][r - 1]] * pagos[i][eventos[i][r - 1]]}{\text{0}}
	\label{eq:1} 
	%\end{equation}
}
\end{minipage}

\vspace{1cm}

\begin{minipage}[t]{19cm}
\aux{calcularRecurso}{recurso: \float, coop: \bool, apuesta: \float, pago: \float, fondoComun: \float, cantidadDeJugadores: \nat}{\float}{%
	res = dividirFondos(apuesta * pago * recurso, coop, fondoComun, cantidadDeJugadores)
}
\end{minipage}

%aqui va un espacio
\vspace{1cm}

\begin{minipage}[t]{19cm}
\aux{dividirFondos}{nuevoRecurso: \float, coop: \bool, fondoComun: \float, cantidadDeJugadores: \nat}{\float}{%
	res = \IfThenElse{coop = true}{$\[\frac{\text{fondoComun}}{\text{cantidadDeJugadores}}\]$}{\text{$\frac{\text{fondoComun}}{\text{cantidadDeJugadores}} + \text{nuevoRecurso}$}}
}
\end{minipage}

\aux{auxiliarSuelto}{parametros}{tipoRes}{expresion}
% \paraTodo{variable}{tipo}{expresion}
% \existe{variable}{tipo}{expresion}
% Pueden tener [unalinea] para que no se divida en varias lineas
\pred{predSuelto}{parametros}{\paraTodo[unalinea]{variable}{tipo}{algo \implicaLuego expresion}}
\pred{predSuelto}{parametros}{\existe[unalinea]{variable}{tipo}{algo \yLuego expresion}}



\end{document}
