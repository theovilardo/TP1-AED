\documentclass[10pt,a4paper]{article}

\usepackage[spanish,activeacute,es-tabla]{babel}
\usepackage[utf8]{inputenc}
\usepackage{ifthen}
\usepackage{listings}
\usepackage{dsfont}
\usepackage{subcaption}
\usepackage{amsmath}
\usepackage[strict]{changepage}
\usepackage[top=1cm,bottom=2cm,left=1cm,right=1cm]{geometry}%
\usepackage{color}%
\newcommand{\tocarEspacios}{%
	\addtolength{\leftskip}{3em}%
	\setlength{\parindent}{0em}%
}

% Especificacion de procs

\newcommand{\In}{\textsf{in }}
\newcommand{\Out}{\textsf{out }}
\newcommand{\Inout}{\textsf{inout }}

\newcommand{\encabezadoDeProc}[4]{%
	% Ponemos la palabrita problema en tt
	%  \noindent%
	{\normalfont\bfseries\ttfamily proc}%
	% Ponemos el nombre del problema
	\ %
	{\normalfont\ttfamily #2}%
	\
	% Ponemos los parametros
	(#3)%
	\ifthenelse{\equal{#4}{}}{}{%
		% Por ultimo, va el tipo del resultado
		\ : #4}
}

\newenvironment{proc}[4][res]{%
	
	% El parametro 1 (opcional) es el nombre del resultado
	% El parametro 2 es el nombre del problema
	% El parametro 3 son los parametros
	% El parametro 4 es el tipo del resultado
	% Preambulo del ambiente problema
	% Tenemos que definir los comandos requiere, asegura, modifica y aux
	\newcommand{\requiere}[2][]{%
		{\normalfont\bfseries\ttfamily requiere}%
		\ifthenelse{\equal{##1}{}}{}{\ {\normalfont\ttfamily ##1} :}\ %
		\{\ensuremath{##2}\}%
		{\normalfont\bfseries\,\par}%
	}
	\newcommand{\asegura}[2][]{%
		{\normalfont\bfseries\ttfamily asegura}%
		\ifthenelse{\equal{##1}{}}{}{\ {\normalfont\ttfamily ##1} :}\
		\{\ensuremath{##2}\}%
		{\normalfont\bfseries\,\par}%
	}
	\renewcommand{\aux}[4]{%
		{\normalfont\bfseries\ttfamily aux\ }%
		{\normalfont\ttfamily ##1}%
		\ifthenelse{\equal{##2}{}}{}{\ (##2)}\ : ##3\, = \ensuremath{##4}%
		{\normalfont\bfseries\,;\par}%
	}
	\renewcommand{\pred}[3]{%
		{\normalfont\bfseries\ttfamily pred }%
		{\normalfont\ttfamily ##1}%
		\ifthenelse{\equal{##2}{}}{}{\ (##2) }%
		\{%
		\begin{adjustwidth}{+5em}{}
			\ensuremath{##3}
		\end{adjustwidth}
		\}%
		{\normalfont\bfseries\,\par}%
	}
	
	\newcommand{\res}{#1}
	\vspace{1ex}
	\noindent
	\encabezadoDeProc{#1}{#2}{#3}{#4}
	% Abrimos la llave
	\par%
	\tocarEspacios
}
{
	% Cerramos la llave
	\vspace{1ex}
}

\newcommand{\aux}[4]{%
	{\normalfont\bfseries\ttfamily\noindent aux\ }%
	{\normalfont\ttfamily #1}%
	\ifthenelse{\equal{#2}{}}{}{\ (#2)}\ : #3\, = \ensuremath{#4}%
	{\normalfont\bfseries\,;\par}%
}

\newcommand{\pred}[3]{%
	{\normalfont\bfseries\ttfamily\noindent pred }%
	{\normalfont\ttfamily #1}%
	\ifthenelse{\equal{#2}{}}{}{\ (#2) }%
	\{%
	\begin{adjustwidth}{+2em}{}
		\ensuremath{#3}
	\end{adjustwidth}
	\}%
	{\normalfont\bfseries\,\par}%
}

% Tipos

\newcommand{\nat}{\ensuremath{\mathds{N}}}
\newcommand{\ent}{\ensuremath{\mathds{Z}}}
\newcommand{\float}{\ensuremath{\mathds{R}}}
\newcommand{\bool}{\ensuremath{\mathsf{Bool}}}
\newcommand{\cha}{\ensuremath{\mathsf{Char}}}
\newcommand{\str}{\ensuremath{\mathsf{String}}}

% Logica

\newcommand{\True}{\ensuremath{\mathrm{true}}}
\newcommand{\False}{\ensuremath{\mathrm{false}}}
\newcommand{\Then}{\ensuremath{\rightarrow}}
\newcommand{\Iff}{\ensuremath{\leftrightarrow}}
\newcommand{\implica}{\ensuremath{\longrightarrow}}
\newcommand{\IfThenElse}[3]{\ensuremath{\mathsf{if}\ #1\ \mathsf{then}\ #2\ \mathsf{else}\ #3\ \mathsf{fi}}}
\newcommand{\yLuego}{\land _L}
\newcommand{\oLuego}{\lor _L}
\newcommand{\implicaLuego}{\implica _L}

\newcommand{\cuantificador}[5]{%
	\ensuremath{(#2 #3: #4)\ (%
		\ifthenelse{\equal{#1}{unalinea}}{
			#5
		}{
			$ % exiting math mode
			\begin{adjustwidth}{+2em}{}
				$#5$%
			\end{adjustwidth}%
			$ % entering math mode
		}
		)}
}

\newcommand{\existe}[4][]{%
	\cuantificador{#1}{\exists}{#2}{#3}{#4}
}
\newcommand{\paraTodo}[4][]{%
	\cuantificador{#1}{\forall}{#2}{#3}{#4}
}

%listas

\newcommand{\TLista}[1]{\ensuremath{seq \langle #1\rangle}}
\newcommand{\lvacia}{\ensuremath{[\ ]}}
\newcommand{\lv}{\ensuremath{[\ ]}}
\newcommand{\longitud}[1]{\ensuremath{|#1|}}
\newcommand{\cons}[1]{\ensuremath{\mathsf{addFirst}}(#1)}
\newcommand{\indice}[1]{\ensuremath{\mathsf{indice}}(#1)}
\newcommand{\conc}[1]{\ensuremath{\mathsf{concat}}(#1)}
\newcommand{\cab}[1]{\ensuremath{\mathsf{head}}(#1)}
\newcommand{\cola}[1]{\ensuremath{\mathsf{tail}}(#1)}
\newcommand{\sub}[1]{\ensuremath{\mathsf{subseq}}(#1)}
\newcommand{\en}[1]{\ensuremath{\mathsf{en}}(#1)}
\newcommand{\cuenta}[2]{\mathsf{cuenta}\ensuremath{(#1, #2)}}
\newcommand{\suma}[1]{\mathsf{suma}(#1)}
\newcommand{\twodots}{\ensuremath{\mathrm{..}}}
\newcommand{\masmas}{\ensuremath{++}}
\newcommand{\matriz}[1]{\TLista{\TLista{#1}}}
\newcommand{\seqchar}{\TLista{\cha}}

\renewcommand{\lstlistingname}{Código}
\lstset{% general command to set parameter(s)
	language=Java,
	morekeywords={endif, endwhile, skip},
	basewidth={0.47em,0.40em},
	columns=fixed, fontadjust, resetmargins, xrightmargin=5pt, xleftmargin=15pt,
	flexiblecolumns=false, tabsize=4, breaklines, breakatwhitespace=false, extendedchars=true,
	numbers=left, numberstyle=\tiny, stepnumber=1, numbersep=9pt,
	frame=l, framesep=3pt,
	captionpos=b,
}

\usepackage{caratula} % Version modificada para usar las macros de algo1 de ~> https://github.com/bcardiff/dc-tex
\usepackage{amsthm}
\usepackage{mathtools}
\usepackage{amssymb}
\usepackage[document]{ragged2e}


\titulo{Trabajo Práctico N1}
\subtitulo{Especificacion y WP}

\fecha{\today}

\materia{Algoritmos y estructura de datos}
\grupo{Grupo Caballo}

\integrante{Nabot, Teo}{996/22}{teonabot@gmail.com}
\integrante{Santillan, Lautaro}{370/22}{lautisantil@gmail.com}
\integrante{Vilardo, Theo}{743/22}{theovilardo@gmail.com}
\integrante{Recchini, Nicolás Gabriel}{37/23}{nicolas.recchini@gmail.com}
% Pongan cuantos integrantes quieran

% Declaramos donde van a estar las figuras
% No es obligatorio, pero suele ser comodo
\graphicspath{{../static/}}

\begin{document}

\maketitle

\section{Especificación}

\vspace{0.6cm}

\subsection{Ejercicio 1}

\begin{proc}{redistribucionDeLosFrutos}{\In recursos : \TLista{\float}, \In cooperan : \TLista{\bool}}{\TLista{\float} \hspace{0.1cm} \{}
	\requiere{|recursos| = |cooperan|}
    	\requiere{todosPostivos(recursos)}
    	\asegura{|res| = |recursos|}
	\asegura{\paraTodo[unalinea]{i}{\ent}{0 \leq i < |res| \yLuego (res[i] = fondoMonetario(recursos, cooperan) / |recursos|)}{}}
\end{proc}
\}

\vspace{1cm}

\aux{fondoMonetario}{\In recursos : \TLista{\float}, \In cooperan : \TLista{\bool}}{\float}{\sum_{i=0}^{n} {\IfThenElse{cooperan[i]}{recursos[i]}{0}}}

\vspace{1cm}

\pred{todosPositivos}{s : \TLista{\float}}{
    \paraTodo[unalinea]{i}{\ent}{0 \leq i < |s| \implicaLuego s[i] > 0}{}
}

\vspace{1cm}

\subsection{Ejercicio 2}

\begin{minipage}{19cm}
\begin{proc}{trayectoriaDeLosFrutosIndividualesALargoPlazo}{\Inout trayectorias : \TLista{\TLista{\mathds{R}}}, \In cooperan : \TLista{\bool}, \In apuestas : \TLista{\TLista{\mathds{R}}}, \In pagos : \TLista{\TLista{\mathds{R}}}, \In eventos : \TLista{\TLista{\mathds{N}}}}{\{}
	\requiere{%
		{trayectoriaInicialValida(trayectorias)} \newline {\land apuestasValidas(apuestas,eventos)} \newline {\land 				\text{mismoTamañoLista(old{(trayectorias)}, apuestas, pagos, eventos, cooperan)}} \newline {\land \text{mismoTamañoSubListas(apuestas, pagos)}} \newline {\land 			\text{eventosValidos(eventos, apuestas)}}
	}
    \requiere{
        \paraTodo[unalinea]{i}{\ent}{0 \leq i < |pagos| \implicaLuego todosPositivos(pagos[i])}{}
    }
	\asegura{%
		|trayectorias| = old{(trayectorias)} \yLuego trayectoriasFinalesValidas(trayectorias, eventos)
	}
	\asegura{%
	\paraTodo[unalinea]{j}{\ent}{0 \leq j < |trayectorias|} \implica {trayectorias[j][0] = old{(trayectorias[j][0])}}
	} 
	\asegura{%
	\paraTodo[unalinea]{j, r}{\ent}{(0 \leq j < |trayectorias|}{\yLuego
	(0 < r < |eventos[j]|))} \implicaLuego trayectorias[j][r] = 
	calcularRecurso(
		trayectorias[j][r - 1],
		cooperan[j],
		pagos[j][eventos[j][r - 1]],
		apuestas[j][eventos[j][r - 1]],
		calcularFondoComun(trayectoria, cooperan, apuestas, pagos, eventos, r)),
		| trayectorias |
	}
\end{proc}
\}

\end{minipage}

\vspace{1cm}

\pred{esMaximo}{s: \TLista{\ent}, m: \ent}{\paraTodo[unalinea]{i}{\ent}{0 \leq i < |s| \implicaLuego }{}}

\vspace{1cm}

\pred{apuestasValidas}{aps,evs: \TLista{\TLista{\ent}}} {\paraTodo[unalinea]{i}{\ent}{0 \leq i < |aps|} \existe[unalinea]{m}{\ent}{esMaximo(evs[i], m)} \implicaLuego {|aps[i]| \geq m}\land{(sumaElementos(aps[i]) = 1)}}

\vspace{1cm}

\aux{sumaElementos}{s: \TLista{\float}}{\ent}{%
	\sum\limits_{k=0}^{| s | - 1} s[k]
}

\vspace{1cm}

\pred{mismoTamañoListas}{s1, s2, s3: \TLista{\TLista{\float}}, s4: \TLista{\TLista{\nat}}, s5: \TLista{\bool}}{|s1| = |s2| \land |s2| = |s3| \land |s3| = |s4| \land |s4| = |s5|}

\vspace{1cm}

\pred{mismoTamañoSublistas}{s1, s2: \TLista{\TLista{\float}}}{\paraTodo[unalinea]{i}{\ent}{0 \leq i < |s1| \land |s1| = |s2| \implicaLuego |s1[i]| = |s2[i]|}}

\vspace{1cm}

\begin{minipage}[t]{20cm}
    \pred{eventosValidos}{s: \TLista{\TLista{\nat}}, l: \TLista{\TLista{\ent}}}{\paraTodo[unalinea]{i}{\ent}{0 \leq i < |s| -1 \implicaLuego |s[i]| = |s[i+1]|} \land {\paraTodo[unalinea]{m}{\ent}{0 \leq m < |s| \land |s| = |l| \implicaLuego {\paraTodo[unalinea]{n}{\ent}{0 \leq n < |s[m]| \implicaLuego 0 \leq s[m][n] < l[m]}}}}}
\end{minipage}


\vspace{1cm}

\pred{trayectoriaInicialValida}{s: \TLista{\TLista{\float}}}{\paraTodo[unalinea]{i}{\ent}{0 \leq i < |s| \implicaLuego |s[i]| = 1 \land s[i][0] > 0}}

\vspace{1cm}

\pred{trayectoriasFinalesValidas}{s1: \TLista{\TLista{\ent}}, s2: \TLista{\TLista{\nat}}}{\paraTodo[unalinea]{i}{\ent}{0 \leq i < |s1| \implicaLuego |s1[i]| = |s2[i]| + 1}}



\begin{minipage}[t]{19cm}
\aux{calcularFondoComun}{trayectorias : \TLista{\TLista{\mathds{R}}}, cooperan : \TLista{\bool}, apuestas : \TLista{\TLista{\mathds{R}}}, pagos : \TLista{\TLista{\mathds{R}}}, eventos : \TLista{\TLista{\mathds{N}}}, r : \nat}{\float}{%
	%\begin{equation}
	\sum\limits_{i=0}^{| trayectorias | - 1} \IfThenElse{cooperan[i] = true}{trayectorias[i][r - 1] * apuestas[i][eventos[i][r - 1]] * pagos[i][eventos[i][r - 1]]}{\text{0}}
	\label{eq:1} 
	%\end{equation}
}
\end{minipage}

\vspace{1cm}

\begin{minipage}[t]{19cm}
\aux{calcularRecurso}{recurso: \float, coop: \bool, apuesta: \float, pago: \float, fondoComun: \float, cantidadDeJugadores: \nat}{\float}{%
	res = dividirFondos(apuesta * pago * recurso, coop, fondoComun, cantidadDeJugadores)
}
\end{minipage}

\vspace{1cm}

\begin{minipage}[t]{19cm}
\aux{dividirFondos}{nuevoRecurso: \float, coop: \bool, fondoComun: \float, cantidadDeJugadores: \nat}{\float}{
	res = \IfThenElse{coop = true}{\frac{\text{fondoComun}}{\text{cantidadDeJugadores}}}{\text{$\frac{\text{fondoComun}}{\text{cantidadDeJugadores}} + \text{nuevoRecurso}$}}
}
\end{minipage}

\subsection{Ejercicio 3}

\begin{proc}{trayectoriaExtrañaEscalera}{\In trayectorias : \TLista{\float}}{\bool \hspace{0.1cm} \{}
	\requiere{|trayectorias| > 0}
	\asegura{%
		res = \existe{i}{\nat}{\hspace{0.4cm} 0 < i < |trayectoria| -1 \hspace{0.2cm} \yLuego \hspace{0.2cm} trayectoria[i] > trayectoria[i+1] \hspace{0.2cm} \wedge \hspace{0.2cm} trayectoria[i] > trayectoria[i-1]} \land (\not\exists{j}:{\nat})(i \not = j \hspace{0.2cm} \land \hspace{0.2cm} 0 < j < |trayectoria|-1 \hspace{0.2cm} \yLuego \hspace{0.2cm} trayectoria[j] > trayectoria[j+1] \hspace{0.2cm} \wedge \hspace{0.2cm} trayectoria[j] > trayectoria[j-1])
	}
\end{proc}
\}
\subsection{Ejercicio 4}

\begin{proc}{individuoDecideSiCooperarONo}{\In individuo: \nat, \In recursos : \TLista{\float}, \Inout cooperan : \TLista{\bool}, \In apuestas : \TLista{\TLista{\float}}, \In pagos : \TLista{\TLista{\float}}, \In eventos : \TLista{\TLista{\nat}}}{\{}
    \newline
    \requiere {
    $0 \leq individuo < |cooperan|$ \newline
    $\land \hspace{0.2cm} apuestasValidas(apuestas,eventos)$ \newline
    $\land \hspace{0.2cm}mismoTamañoListas2(recursos,apuestas,pagos,eventos,cooperan)$ \newline
    $\land \hspace{0.2cm} mismoTamañoSublistas(apuestas,pagos)$ \newline
    $\land \hspace{0.2cm} eventosValidos(eventos,apuestas)$}
    \asegura{cooperan[individuo] = convieneCooperar(individuo, recursos, old{(cooperan)}, apuestas, pagos, eventos)}
    \asegura{{|old(cooperan)|=|cooperan|} \newline \land{\paraTodo[unalinea]{i}{\ent}{0\leq i < |cooperan| \land i \neq individuo \implicaLuego old(cooperan[i]) = cooperan[i]}} }
\end{proc}

\begin{minipage}[t]{17cm}
	\pred{convieneCooperar}{ind: \nat, rs: \TLista{\float}, coop: \TLista{\bool}, as: \TLista{\TLista{\float}}, ps: \TLista{\TLista{\float}}, es: \TLista{\TLista{\nat}}}{%
			\existe{s1, s2}{\TLista{\nat}}{trayectoriaCorrecta(s1, ind, rs, setAt(coop, ind, true), as, ps, es) \land \\
			trayectoriaCorrecta(s2, ind, rs, setAt(coop, ind, false), as, ps, es \\
			\implica \IfThenElse{s1[ind][|s1|-1] \geq s2[ind][|s2|-1]}{true}{false})}	
		}
	\end{minipage}
\vspace{1cm}

\begin{minipage}[t]{20cm}
    \pred{trayectoriaCorrecta}{s: \TLista{\TLista{\float}}, ind: \nat, rs: \TLista{\float}, coop: \TLista{\bool}, as: \TLista{\TLista{\float}}, ps: \TLista{\TLista{\float}}, es: \TLista{\TLista{\nat}}}{%
	|s[ind]| = |es[ind]| \yLuego s[ind][0]  =  rs[ind] \land \paraTodo{j}{\ent}{0 < j \leq |s[ind]-1| \implica s[ind][j] = calcularRecurso(s[ind][j-1], coop, as[ind][es[ind][j-1]], ps[ind][es[ind][j-1]], calcularFondoComun(s, coop, as, ps, es, j-1), |s|) }
}
\end{minipage}
\vspace{1cm}
	
\pred{mismoTamañoListas2}{s1: \TLista{\float}, s2, s3: \TLista{\TLista{\float}}, s4: \TLista{\TLista{\nat}}, s5: \TLista{\bool}}{|s1| = |s2| \land |s2| = |s3| \land |s3| = |s4| \land |s4| = |s5|}

\subsection{Ejercicio 5}


\begin{proc}{individuoActualizaApuesta}{\In individuo: \nat, \In recursos : \TLista{\float}, \In cooperan : \TLista{\bool}, \Inout apuestas : \TLista{\TLista{\float}}, \In pagos : \TLista{\TLista{\float}}, \In eventos : \TLista{\TLista{\nat}}}{\{}
        \newline
        \requiere {
            ${0 \leq individuo < |apuestas|}$ \newline \land{apuestasValidas(apuestas,eventos)} \newline \land{mismoTamañoListas2(recursos,apuestas,pagos,eventos,cooperan)} \newline \land{mismoTamañoSublistas(apuestas,pagos)} \newline \land{eventosValidos(eventos,apuestas)}  \newline \land{todosPositivos(recursos) \newline \land
        \paraTodo[unalinea]{i}{\ent}{0 \leq i < |pagos| \implicaLuego todosPositivos(pagos[i])}{}
    }}
    \begin{minipage}{18cm}
    \asegura{
            \paraTodo{s}{\TLista{\float}}{\hspace{0.35cm} apuestasValidas(setAt(apuestas, individuo, s), eventos) \implicaLuego esMejorApuesta(apuestas, setAt(apuestas, individuo, s), individuo, recursos, cooperan, pagos, eventos)}
        }
        
    \end{minipage}    
	    \asegura{ \hspace{0.2cm}
            |old{(apuestas)}|=|apuestas| \hspace{0.2cm} \yLuego \\ \vspace{0.1cm} \hspace{0.2cm} {\paraTodo[unalinea]{i}{\ent}{$0\leq i < |apuestas|$ \implicaLuego {\paraTodo[unalinea]{j}{\ent}{0\leq j < |apuestas[i]| \land i \neq individuo} \\
            \hspace{0.3cm} \implicaLuego \hspace{0.2cm} {apuestas[i][j]=old(apuestas[i][j])}\\}}}
        }
\end{proc}
\}

\vspace{0.6cm}

%\begin{minipage}[t]{19cm}
    \pred{esMejorApuesta}{as_1: \TLista{\TLista{\float}}, as_2: \TLista{\TLista{\float}},ind: \ent, rs : \TLista{\float}, coop: \TLista{\bool}, ps: \TLista{\TLista{\float}}, es: \TLista{\TLista{\nat}}}{%
    \existe[unalinea]{ts_1,ts_2}{\TLista{\TLista{\float}}} {trayectoriaCorrecta(ts_1, ind, rs, coop, as_1, ps, es) \\ \hspace{0.2cm} \hspace{0.4cm} \land \hspace{0.2cm} trayectoriaCorrecta(ts_2, ind, rs, coop, as_2, ps, es)
    \yLuego ts_1[ind][|ts_1[ind]|-1] \geq ts_2[ind][|ts_2[ind]|-1]}}
%\end{minipage}
 
\section{Demostración de correctitud}

Se usaran las abreviaturas $rec = recurso$, $ap = apuesta$, $pa = pago$, $evs = eventos$.\\
\begin{equation*}
    P \equiv ap_c + ap_s = 1 \land pa_c > 0 \land pa_s > 0 \land ap_c > 0 \land ap_s > 0 \land recurso > 0
\end{equation*}
\begin{equation*}
    Q \equiv  res = rec (ap_c pa_c)^{\# apariciones(eeventos, T)}(ap_s pa_s)^{\#apariciones(eventos, F)}
\end{equation*}
Se propone el siguiente requiere (Pc) y asegura (Qc) del ciclo:
\begin{equation*}
    Pc \equiv res=recursos \land i=0 \land ap_c + ap_s = 1 \land pa_c > 0 \land pa_s > 0 \land ap_c > 0 \land ap_s > 0 \land recurso > 0
\end{equation*}
\begin{equation*}
    Qc \equiv res = rec (ap_c pa_c)^{\# apariciones(eeventos, T)}(ap_s pa_s)^{\#apariciones(eventos, F)} \equiv Q
\end{equation*}
Se demuestra que el Pc es correcto para iniciar el ciclo con la siguiente demostracion: \newline P \implica $wp(res:=recursos;i:=0,Pc)$ \newline $wp(res:=recursos;i:=0,Pc) \equiv wp(res:=recursos,wp(i:=0,Pc))$ \newline $wp(i:=0,Pc) \equiv def(0) \yLuego Pc_{0}^i \equiv res=recursos \land ap_c + ap_s = 1 \land pa_c > 0 \land pa_s > 0 \land ap_c > 0 \land ap_s > 0 \land recurso > 0$ \newline $wp(res:=recursos,wp(i:=0,Pc)) \equiv wp(res:=recursos,Pc_{0}^i) \equiv def(recursos) \yLuego (Pc_{0}^i)_{recursos}^{res} \equiv ap_c + ap_s = 1 \land pa_c > 0 \land pa_s > 0 \land ap_c > 0 \land ap_s > 0 \land recurso > 0$ \newline
Quiero probar que \newline P \implica $wp(res:=recursos;i:=0,Pc)$ \newline \equiv P \implica P \equiv true \newline

Para probar la correctitud parcial del ciclo, se propone el invariante:
\begin{equation*}
	I \equiv 0 \leq i \leq |evs| \wedge res = rec (ap_c pa_c)^{\# apariciones(subseq(evs, 0, i), T)}(ap_s pa_s)^{\#apariciones(subseq(evs, 0, i), F)}
\end{equation*}	

Así, por el teorema del invariante, el ciclo $while(B) \, do \, S$ es parcialmente correcto respecto $(P_c, Q_c) $ sii:

\begin{enumerate} \setlength\itemsep{0cm}
	\item $P_c \implica I$
	\item $\{I \wedge B \} S \{I\}$
	\item $\{I \wedge \neg B \} \implica Q_c$
\end{enumerate}

Para probar 1. tenemos $P_c = i=0 \wedge res=rec \wedge ap_c + ap_s = 1 \wedge p_c > 0 \wedge p_s > 0 \wedge ap_c > 0 \wedge ap_s > 0 \wedge rec > 0$ y queremos ver que \\ $0 \leq i < |evs| \wedge res = rec (ap_c pa_c)^{\# apariciones(subseq(evs, 0, i), T)}(ap_s pa_s)^{\#apariciones(subseq(evs, 0, i), F)}$

\begin{proof*}
    $P_c \implica  i = 0 \implica 0 \leq i \leq |evs| \\
    \implica subseq(evs, 0, i) = subseq(evs, 0, 0) = \lvacia \\
    \implica \#(subseq(evs, 0, i), T) = \#(subseq(evs, 0, i), F) = 0  $\\
    Como cualquier número elevado a cero es 1, tenemos: \\
    $(ap_c pa_c)^{\#(subseq(evs, 0, i), T)} = (ap_s pa_s)^{\#(subseq(evs, 0, i), F)} = 1 \\
	\implica rec * (ap_c pa_c)^{\#(subseq(evs, 0, i), T)} * (ap_s pa_s)^{\#(subseq(evs, 0, i), F)} = rec $\\
    Además, $P_c$ \implica $res = rec$. Luego, $res = rec * (ap_c pa_c)^{\#(subseq(evs, 0, i), T)} * (ap_s pa_s)^{\#(subseq(evs, 0, i), F)} \wedge 0 \leq i \leq |evs|$, que es el invariante.
\end{proof*}

Luego, para probar 2., tenemos que ver que la tripla de Hoare $\{I \wedge B \} S \{I\}$ es válida, es decir, probar que $\{I \wedge B \} \implica wp(S, I) $.\\

\begin{proof*}
    Tenemos que $S \equiv S1; S2$, con \\
    $S1 \equiv \IfThenElse{eventos[i]}{res:=res * ap_c * p_c}{res:=res*ap_s*p_s}$\\
    $S2 \equiv i:= i+1$\\
    Así, podemos usar el axioma de secuencia de programas para calcular $wp(S,I)$\\
	$wp(S, I) \equiv wp(S1; S2, I) \equiv wp(S1, wp(S2, I))$.
	Caclulemos primero $wp(S2, I)$. \\ 
	$wp(S2, I) \equiv def(i+1) \yLuego I_{i+1}^i \\
    \equiv I_{i+1}^i \\
	\equiv 0 \leq i+1 \leq |evs| \wedge res = rec *  (ap_c * pa_c)^{\#(subseq(evs, 0, i+1), true)} * (ap_s * pa_s)^{\#(subseq(evs, 0, i+1), false)}$\\
	$wp(S1, wp(S2, I)) \equiv wp(S1, I_{i+1}^i) \equiv def(evs[i]) \yLuego ((evs[i] \wedge wp(res:=res * ap_c * pa_c, I_{i+1}^i)) \vee (evs[i] = false \wedge wp(res:=res * ap_s * pa_s, I_{i+1}^i)) ) $\\
	$\equiv 0 \leq i < |evs| \yLuego ((evs[i] = true \wedge wp_1) \vee (evs[i] = false \wedge wp_2))$, donde \\
	$wp_1 \equiv wp(res:= res* ap_c * pa_c, I_{i+1}^i)$ y $wp_2 \equiv wp(res:= res*ap_s *pa_s, I_{i+1}^i)$. 
	Procedemos a calcular $wp_1, wp_2$\\
	$\equiv def (res*ap_s*pa_s) \yLuego (I_{i+1}^i)_{res*ap_c*pa_c}^{res} \\
 \equiv (I_{i+1}^i)_{res*ap_c*pa_c}^{res} \\
 \equiv 0\leq i+1 \leq |evs| \wedge res * ap_c * pa_c = rec *(ap_ * pa_c)^{\#(subseq(evs, 0, i+1), true)} * (ap_s * pa_s)^{\#(subseq(evs, 0, i+1), false)}$\\
	$\equiv 0\leq i+1 \leq |evs| \wedge res = rec *(ap_c * pa_c)^{\#(subseq(evs, 0, i+1), true) - 1} * (ap_s * pa_s)^{\#(subseq(evs, 0, i+1), false)}$\\
	Véase que se pudo pasar dividiendo $ap_c * pa_c$ porque $ap_c * pa_c > 0 \implica ap_c * pa_c \not= 0$ \\
	Similarmente, para $wp_2$, calculamos:
	$wp_2 \equiv def (res*ap_s*pa_s) \yLuego (I_{i+1}^i)_{res*ap_s*pa_s}^{res} \\
 \equiv (I_{i+1}^i)_{res*ap_s*pa_s}^{res} \\
 \equiv 0\leq i+1 \leq |evs| \wedge res * ap_s * pa_s = rec *(ap_c * pa_c)^{\#(subseq(evs, 0, i+1), true)} * (ap_s * pa_s)^{\#(subseq(evs, 0, i+1), false)}$\\
	$\equiv 0\leq i+1 \leq |evs| \wedge res = rec *(ap_c * pa_c)^{\#(subseq(evs, 0, i+1), true)} * (ap_s * pa_s)^{\#(subseq(evs, 0, i+1), false) - 1}$\\
	Ya calculamos $wp_1, wp_2$. Reemplazando $wp_1, wp_2$ respectivamente, llegamos a \\
	$wp(S, I) \equiv 0 \leq i < |evs| \yLuego ((evs[i] = true \wedge 0\leq i+1 \leq |evs| \wedge res = rec *(ap_c * pa_c)^{n - 1} * (ap_s * pa_s)^m) \vee $\\
	$(evs[i] = false) \wedge 0\leq i+1 \leq |evs| \wedge res = rec *(ap_c * pa_c)^{n} * (ap_s * pa_s)^{m - 1})$\\
	con $n = \#apariciones(subseq(evs, 0, i+1), true) \wedge m = \#apariciones(subseq(evs, 0, i+1), false)$\\
	Sacando $0 \leq i+1 \leq |evs|$ como factor común de la disyunción y observando que $0 \leq i < |evs| \implica 0 \leq i + 1 \leq |evs|$, resulta:
	\begin{equation*}
		wp(S, I) \equiv 0 \leq i < |evs|  \yLuego ((evs[i] \wedge res = rec *(ap_c * pa_c)^{n - 1} * (ap_s * pa_s)^m) \vee 
		(\neg evs[i]) \wedge res = rec *(ap_c * pa_c)^n * (ap_s * pa_s)^{m - 1})
	\end{equation*}
    \begin{minipage}[t]{19cm}
        \begin{itemize}
            \item{Caso eventos[i + 1] = true: \\
            \implica $n = \#(subseq(evs, 0, i+1), true) = \#(subseq(evs, 0 ,i), true) + 1 $\\
            Y, además, \\
            $m = \#(subseq(evs, 0, i+1), false) = \#(subseq(evs, 0, i), false)$ \\
            \implica $rec *(ap_c * pa_c)^{n - 1} * (ap_s * pa_s)^m) = rec *(ap_c * pa_c)^{\#(subseq(evs, 0 ,i), true)} * (ap_s * pa_s)^{\#(subseq(evs, 0, i), false))} $ \\
            \hspace{0.1cm}
            \text{Reemplazando en $wp(S,I)$, tenemos}\\
            $wp(S,I) \equiv 0 \leq i < |evs| \yLuego ((evs[i] \land res = rec *(ap_c * pa_c)^{\#(subseq(evs, 0 ,i), true)} * (ap_s * pa_s)^{\#(subseq(evs, 0, i), false))} \lor (\neg evs[i] \land res = rec * (ap_c * pa_c)^{\#(subseq(evs, 0 ,i), true) + 1} * (ap_s * pa_s)^{\#(subseq(evs, 0, i), false) - 1}$
            \\
            Concentrémonos en la primera parte de la disyunción, es decir: \\
            $(evs[i] \land res = rec *(ap_c * pa_c)^{\#(subseq(evs, 0 ,i), true)} * (ap_s * pa_s)^{\#(subseq(evs, 0, i), false))}$
            Tenemos que $B \implica evs[i] \land I \implica res = rec *(ap_c * pa_c)^{\#(subseq(evs, 0 ,i), true)} * (ap_s * pa_s)^{\#(subseq(evs, 0, i), false))}$, ya que son los mismo. Entonces, como tenemos una disyunción, la misma es True. \\ Además, $I \land B \implica  0 \leq i < |evs|  $ Luego, resulta $I\land B \implica wp(S,I)$. 
        \end{itemize}
    \end{minipage}
    
    \begin{minipage}[t]{19cm}
        \begin{itemize}
        \item{Caso eventos[i + 1] = false: \\
            \implica $n = \#(subseq(evs, 0, i+1), true) = \#(subseq(evs, 0 ,i), true)$\\
            Y, además, \\
            $m = \#(subseq(evs, 0, i+1), false) = \#(subseq(evs, 0, i), false)$  + 1\\
    }
     Equivalentemente al caso anterior, llegamos a : \\
     $wp(S,I) \equiv 0 \leq i < |evs| \yLuego ((evs[i] \land res = rec *(ap_c * pa_c)^{\#(subseq(evs, 0 ,i), true) - 1} * (ap_s * pa_s)^{\#(subseq(evs, 0, i), false)) + 1} \lor (\neg evs[i] \land res = rec * (ap_c * pa_c)^{\#(subseq(evs, 0 ,i), true)} * (ap_s * pa_s)^{\#(subseq(evs, 0, i), false)}$ \\ Y, con el mismo razonamiento, concentrándonos en la segunda parte de la disyunción, vemos que $I\land B \implica wp(S,I)$ .
    \end{itemize}
    \end{minipage}

\vspace{1cm}
    \\Luego, para probar 3., tenemos

\begin{proof*}
	$ \{\neg B \wedge  I\} \implica i \geq |evs| \wedge 0 \leq i \leq |evs|	\implica i = |evs| \\$
	Luego, $I \wedge i = |evs| \implica res = rec * (ap_c pa_c)^{\#(subseq(evs, 0, i), T)} * (ap_s pa_s)^{\#(subseq(evs, 0, i), F)} \\
	\equiv res = rec * (ap_c pa_c)^{\#(subseq(evs, 0, |evs|), T)} * (ap_s pa_s)^{\#(subseq(evs, 0, |evs|), F)} \\ \equiv res = rec * (ap_c pa_c)^{\#(evs, T)} * (ap_s pa_s)^{\#(evs, F)} \\ \equiv Q_c$ 
\end{proof*}

\vspace{1cm}


    

    \vspace{1cm}
    Vemos que se cumple 4: 
    $ \{I \land B \land fv = v_0\} \hspace{0.1cm} S \hspace{0.1cm} \{fv < v_0\}$
\vspace{0.5}

Que es pedir que: $I \land B \land fv = v_0 \implica wp(S, fv < v_0)$
     
Luego,\hspace{0.2cm} \text{quiero ver que vale:} \text{ $wp(S, fv < v_0) \equiv$ ${wp(S1, wp(S2, fv < v_0))}$} \\
    \vspace{0.2cm}
    \begin{minipage}[t]{18cm}
        \begin{itemize}
        \item {$wp(S2, fv < v_0) \equiv def(i + 1) \yLuego |eventos| -i-1 < v_0$}
        \item {$wp(S1, wp(S2, fv < v_0)) 
        \equiv wp(S1, |eventos|-i-1 < v_0) \\
        \equiv def(eventos[i]) \yLuego (eventos[i] = true \land wp(res \coloneq res * apuesta[c] * pagos[c], |eventos| -i-1 < v_0)) \lor (eventos[i] = false \land wp(res \coloneq res * apuestas[s] * pagos[s], |eventos|-i-1 < v_0)) \\$
        Luego, \\
        $def(eventos[i]) \equiv 0 \leq i < |eventos|$ \\ 
        $wp(res \coloneq res * apuesta[c] * pagos[c], |eventos| -i-1 < v_0)) \equiv def(res * apuesta[c] * pagos[c]) \yLuego {(|eventos|-i-1 < v_0)}^{res}_{res * apuesta[c] * pagos[c]}\equiv |eventos|-i-1 < v_0$ \\
        
        $
        \equiv 0 \leq i < |eventos| \land |eventos| -i-1 < v_0 \hspace{0.1cm} \yLuego \hspace{0.1cm} (eventos[i] = true \lor eventos[i] = false) \\
        \equiv 0 \leq i < |eventos| \lor |eventos| \land |eventos| -i-1 < v_0 \equiv wp(S, fv < v_0)$}
        \item {$I \land B \land fv = v_0 \equiv 0 \leq i < |eventos| \land res = recurso * (apuesta[c]*pagos[c])^{n} * (apuestas[s]* pagos[s])^{m} \land |eventos| - i = v_0$}
        \begin{itemize}
              \item {$0 \leq i < |eventos| \implica 0 \leq i < |eventos| \equiv true$}
              \item {$|eventos| - i = v_0 \implica |eventos| -i-1 < v_0 \equiv v_0 - 1 < v_0 \equiv -1 < 0 \equiv true$}
        \end{itemize}
        \vspace{0.5cm}
        Entonces vale$ \{I \land B \land fv = v_0\} \hspace{0.1cm} S \hspace{0.1cm} \{fv < v_0\}$
    \end{itemize}
    \end{minipage}
    
\vspace{1cm}

Por ultimo, queremos ver que el ciclo termina. Para esto, queremos probar 5:
    \begin{itemize}
        \item $I \land fv \leq 0 \implica \neg B$
    \end{itemize}

\begin{itemize}
    \vspace{0.2cm}
    \item fv = ${|eventos|}$ - i: ${|eventos|}$ es fijo e ``i'' va aumentando de a 1 en la implementacion, por lo que fv decrece y termina.
    \item $\neg  B \equiv$ \hspace{0.2cm} i \hspace{0.2cm} $\geq$
    \hspace{0.2cm}
    ${|eventos|}$
    \item {$fv \leq 0 \equiv$ \hspace{0.1cm} ${|eventos|} - i$ $\leq 0 \equiv$ \hspace{0.1cm} ${|eventos|} \leq i$}, \hspace{0.2cm} entonces,
    \vspace{0.2cm} $fv \leq 0 \implica \neg B \equiv$ \vspace{0.2cm} ${|eventos|} \leq i \implica i \geq {|eventos|} \equiv true$
    \begin{itemize}
              \item En este caso no fue necesario mirar la informacion del invariante \\
              
        \end{itemize}
    \end{itemize}

    Queda demostrado que el ciclo es correcto y termina.






\vspace{1cm}

\section{Anexo}

\begin{minipage}[t]{18cm}
    Durante el desarrollo del trabajo práctico nos encontramos con varios desafíos, entre ellos, el más complicado de superar fue especificar de la manera esperada. Durante la primera etapa de la carrera nos la pasamos calculando y programando, es por eso que naturalmente nos inclinamos a pensar de esa manera. Es un regalo que nos dio esta materia el poder de la abstracción, el cual confiamos haber adquirido.

    Adjuntamos a continuación, nuestra primera versión del ejercicio 1.4 La misma no es parte de la entrega final si no un simple testigo del progreso que hicimos como especificadores.
\end{minipage}
\vspace{1cm}
\begin{proc}{individuoDecideSiCooperarONo}{\In individuo: \nat, \In recursos: \TLista{\float}, \Inout cooperan: \TLista{\bool}, \In apuestas:
\TLista{\TLista{\float}}, \In pagos: \TLista{\TLista{\float}}, \In eventos: \TLista{\TLista{\nat}}}{\{}
	\requiere {
    		$0 \leq individuo < |cooperan|$ \newline
    		$\land \hspace{0.2cm} apuestasValidas(apuestas,eventos)$ \newline
    		$\land \hspace{0.2cm}mismoTamañoListas2(recursos,apuestas,pagos,eventos,cooperan)$ \newline
    		$\land \hspace{0.2cm} mismoTamañoSublistas(apuestas,pagos)$ \newline
    		$\land \hspace{0.2cm} eventosValidos(eventos,apuestas)$
        }\} \\
      		%puede faltar alguno
	\asegura{%
		cooperan[individuo] = convieneCooperar(individuo, recursos, old{(trayectorias)}, apuestas, pagos, eventos)
	}
\end{proc}
\}

%aqui va un espacio
\vspace{1cm}

\begin{minipage}[t]{17cm}
\aux{convieneCooperar}{ind: \nat, rs: \TLista{\float}, coop: \TLista{\bool}, as: \TLista{\TLista{\float}}, ps: \TLista{\TLista{\float}}, es: \TLista{\TLista{\nat}}}{\bool}{%
	res = \IfThenElse{calculaRecursoFinalCooperando(|es[ind]| - 1, rs, ind, coop, as, es, ps, |rs|) \geq calculaRecursoFinalNoCooperando(|es[ind]| - 1, rs, ind, coop, as, es, ps, |rs|)}{true}{false}
}
\end{minipage}

%aqui va un espacio
\vspace{1cm}

\begin{minipage}[t]{19cm}
\aux{calculaRecursoFinalCooperando}{tiempo: \nat, ind: \nat, rs: \TLista{\float}, coop: \TLista{\bool}, as: \TLista{\TLista{\float}}, ps: \TLista{\TLista{\float}}, es: \TLista{\TLista{\nat}}, cantidadDeJugadores: \nat}{\float}{%
res = \frac{\text{fondoMonetarioComún(tiempo, fondoInicial(ind, coop, rs, as, ps, es), coop, ps, es, as)}}{\text{cantidadDeJugadores}}}
\end{minipage}

%aqui va un espacio
\vspace{1cm}

\begin{minipage}[t]{19cm}
\aux{calculaRecursoFinalNoCooperando}{tiempo: \nat, ind: \nat, rs: \TLista{\float}, coop: \TLista{\bool}, as: \TLista{\TLista{\float}}, ps: \TLista{\TLista{\float}}, es: \TLista{\TLista{\nat}}}{\float}{%
	res = \sum\limits_{i=0}^{| tiempo | - 1} \text{fondoMonetarioComún(tiempo, fondoInicial(i, coop, rs, as, ps, es), coop, as, ps, es)}
}
\end{minipage}

%aqui va un espacio
\vspace{1cm}

\begin{minipage}[t]{17cm}
\aux{fondoMonetarioComun}{tiempo: \nat, fondoInicial: \float, coop: \TLista{\bool}, as: \TLista{\TLista{\float}}, ps: \TLista{\TLista{\float}}, es: \TLista{\TLista{\nat}}}{\float}{%
	res = \prod_{i=1}^{tiempo -1} 
	\left(\sum\limits_{j=0}^{| eventos | - 1} \IfThenElse{coop[j] = true}{\frac{(as[j][es[j][i]] * ps[j][es[j][i]])}{n^(tiempo - 1)}}{\text{0}}\right) * fondoInicial
}
\end{minipage}

%aqui va un espacio
\vspace{1cm}

\begin{minipage}[t]{19cm}
\aux{fondoInicial}{ind: \nat, rs: \TLista{\float}, coop: \TLista{\bool}, as: \TLista{\TLista{\float}}, ps: \TLista{\TLista{\float}}, es: \TLista{\TLista{\nat}}}{\float}{%
	res = \sum\limits_{j=0}^{| rs | - 1} \IfThenElse{cooperan[ind] = true}{(rs[j] * as[j][es[j][0]] * ps[j][es[j][0]])}{\text{0}}
}
\end{minipage}

%aqui va un espacio
\vspace{1cm}

“Un gran poder, conlleva una gran responsabilidad; Y una gran especificación, una gran cantidad de predicados”. \\
-Grupo Caballo

\end{document}
